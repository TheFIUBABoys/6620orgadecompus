\documentclass{article}
\usepackage{latexsym}
\usepackage[utf8]{inputenx}
\usepackage[spanish]{babel}
\usepackage{graphicx}
\usepackage{anysize}
\usepackage{amsmath}
\usepackage{amssymb}
\usepackage{float}
\setlength{\skip\footins}{5cm}
\usepackage{lscape}
\usepackage{verbatim}
\usepackage{moreverb}
\usepackage{url}
\usepackage{enumitem}
\usepackage{multicol}
\let\verbatiminput=\verbatimtabinput
\usepackage[nottoc,numbib]{tocbibind}
\setcounter{tocdepth}{4}
\setcounter{secnumdepth}{4}

\marginsize{2cm}{2cm}{.5cm}{3cm} 

\begin{document}

\begin{titlepage}

\newcommand{\HRule}{\rule{\linewidth}{0.5mm}} % Defines a new command for the horizontal lines, change thickness here

\center % Center everything on the page
 
%----------------------------------------------------------------------------------------
%	HEADING SECTIONS
%----------------------------------------------------------------------------------------

\textsc{\LARGE Universidad De Buenos Aires}\\[1.5cm] % Name of your university/college
\textsc{\Large Facultad De Ingeniería}\\[0.5cm] % Major heading such as course name
\textsc{\large 66.20 Organización De Computadoras}\\[0.5cm] % Minor heading such as course title

%----------------------------------------------------------------------------------------
%	TITLE SECTION
%----------------------------------------------------------------------------------------

\HRule \\[0.4cm]
{ \huge \bfseries Trabajo Práctico 0}\\[0.4cm] % Title of your document
\HRule \\[1.5cm]
 
%----------------------------------------------------------------------------------------
%	AUTHOR SECTION
%----------------------------------------------------------------------------------------

% If you don't want a supervisor, uncomment the two lines below and remove the section above
\Large \emph{Integrantes:}\\
Gonzalo \textsc{Beviglia} - 93144\\ % Your name
Federico \textsc{Quevedo} - 93159\\ % Your name
Damián \textsc{Manoff} - 93169\\[5cm] % Your name

%----------------------------------------------------------------------------------------
%	LOGO SECTION
%----------------------------------------------------------------------------------------

\includegraphics[scale=0.5]{UBA.jpg}\\[1cm] % Include a department/university logo - this will require the graphicx package

%----------------------------------------------------------------------------------------
%	DATE SECTION
%----------------------------------------------------------------------------------------

{\large \text \em {10 de Septiembre de 2013}}\\[3cm] % Date, change the \today to a set date if you want to be precise
 
%----------------------------------------------------------------------------------------

\vfill % Fill the rest of the page with whitespace

\end{titlepage}

\section{Diseño e implementación}

Uno de los principales limitantes de las soluciones que pudimos implementar era el largo del archivo a invertir, ya que este podia llegar a ocupar mucho lugar en la memoria. Buscamos mejorar esta situación leyendo e invirtiendo de a un archivo por vez, para evitar tener mas de uno en memoria. También evitamos tener que declarar otro espacio de memoria del mismo tamaño del archivo original, donde iria la cadena invertida, haciendo una inversion in situ de la cadena, es decir, sobre la misma memoria reservada a donde se cargo el archivo. Esto ademas de ahorrarnos duplicar la memoria usada nos ahorra el tiempo de reservar la misma. Sobre la cadena original se realizan swaps espejados hasta llegar a la cadena invertida.

\section{Performance}

La performance se evaluó invirtiendo el libro ``El Príncipe'' de Nicolás Maquiavelo, y se comparó
la performance del comando realizado para este trabajo práctico con la del comando unix \emph{rev}.
El tamaño de dicho texto en formato de texto plano es de 305864 bytes (298KB).

Los tiempos se midieron utilizando el comando Unix \emph{time}.

\subsection{Tiempo \emph{ownRev}}

\begin{verbatim}
real	0m0.539s

user	0m0.008s

sys	0m0.016s
\end{verbatim}



\subsection{Tiempo Unix \emph{rev}}

\begin{verbatim}
real	0m0.540s

user	0m0.012s

sys	0m0.024s
\end{verbatim}

\section{Compilación del programa}

Para el compilado del programa hicimos el siguiente makefile:

\begin{verbatim}
CFLAGS=-Wall -pedantic -std=c99 -g
MEMFLAGS=valgrind --leak-check=full --track-origins=yes -v
CC=gcc
INPUT=ownRev.c
MIDDLEFILE=ownRev.o
EXEC=ownRev
TESTSCRIPT=TestFiles/tests.sh

all: $(EXEC) clean

.SILENT:
$(MIDDLEFILE):
	$(CC) $(CFLAGS) $(INPUT) -c

.SILENT:
$(EXEC): $(MIDDLEFILE)
	$(CC) $(CFLAGS) $(MIDDLEFILE) -o $(EXEC)

.SILENT:	
run:
	./$(EXEC)

.SILENT:
runTests:
	./$(TESTSCRIPT)

.SILENT:
memCheck:
	$(MEMFLAGS) ./$(EXEC) TestFiles/test TestFiles/test1 TestFiles/test2

.SILENT:
clean:
	rm *.o

.SILENT:
cleanExec:
	rm $(EXEC)

\end{verbatim}

La ejecución normal de este make file produce el archivo ejecutable y ademas elimina los intermediarios.

Se puede tambien llamar pasando como parametro el nombre del archivo intermediario para generarlo, o el nombre del ejecutable, que realizara lo mismo que la ejecución por defecto pero sin eliminar el intermediario.

Para corroborar que no se estuviera perdiendo memoria tambien incluimos el parametro \(memCheck\) que corre el programa con valgrind informando si hubo o no alguna perdida.

Desde el mismo makefile tambien incluimos la posibilidad de correr las pruebas, y por ultimo, la de eliminar los archivos generados, tanto intermediarios como programa final.

\section{Pruebas}

Para las pruebas hicimos el siguiente script de bash:

\begin{verbatim}
#!/bin/bash
FILES="TestFiles/test
TestFiles/test1
TestFiles/test2"

for f in $FILES
do
   ./ownRev $f > auxRev
   ./ownRev auxRev > doubleRev
   DIFF=$(diff $f doubleRev)
   if [ "$DIFF" != "" ] 
   then
      echo "[ERROR] $f was not reversed correctly"
   else
      echo "[OK] $f was reversed correctly"
   fi
   rm auxRev
   rm doubleRev
done
\end{verbatim}

Se toman lineas de un archivo, en este caso llamado test, y se las invierte dos veces. Si la inversion se realizo correctamente no deberia haber diferencias entre el texto introducido y el resultante.

\section{C\'odigo Assembly MIPS}
A continuaci\'on se detallará el c\'odigo assembly para la arquitectura MIPS de nuestro programa en C
\subsection{C\'odigo assembly}
\begin{verbatim}
		.file	1 "tp0v2.c"
	.section .mdebug.abi32
	.previous
	.abicalls
	.rdata
	.align	2
$LC0:
	.ascii	"-v\000"
	.align	2
$LC1:
	.ascii	"--version\000"
	.align	2
$LC2:
	.ascii	"Version 1.0.0\n\000"
	.align	2
$LC3:
	.ascii	"-h\000"
	.align	2
$LC4:
	.ascii	"--help\000"
	.align	2
$LC5:
	.ascii	"Usage\n\000"
	.text
	.align	2
	.globl	checkOption
	.ent	checkOption
checkOption:
	.frame	$fp,48,$ra		# vars= 8, regs= 3/0, args= 16, extra= 8
	.mask	0xd0000000,-8
	.fmask	0x00000000,0
	.set	noreorder
	.cpload	$t9
	.set	reorder
	subu	$sp,$sp,48
	.cprestore 16
	sw	$ra,40($sp)
	sw	$fp,36($sp)
	sw	$gp,32($sp)
	move	$fp,$sp
	sw	$a0,48($fp)
	lw	$a0,48($fp)
	la	$a1,$LC0
	la	$t9,strcmp
	jal	$ra,$t9
	beq	$v0,$zero,$L19
	lw	$a0,48($fp)
	la	$a1,$LC1
	la	$t9,strcmp
	jal	$ra,$t9
	bne	$v0,$zero,$L18
$L19:
	la	$a0,$LC2
	la	$t9,printf
	jal	$ra,$t9
	li	$v0,1			# 0x1
	sw	$v0,24($fp)
	b	$L17
$L18:
	lw	$a0,48($fp)
	la	$a1,$LC3
	la	$t9,strcmp
	jal	$ra,$t9
	beq	$v0,$zero,$L22
	lw	$a0,48($fp)
	la	$a1,$LC4
	la	$t9,strcmp
	jal	$ra,$t9
	bne	$v0,$zero,$L20
$L22:
	la	$a0,$LC5
	la	$t9,printf
	jal	$ra,$t9
	li	$v0,1			# 0x1
	sw	$v0,24($fp)
	b	$L17
$L20:
	sw	$zero,24($fp)
$L17:
	lw	$v0,24($fp)
	move	$sp,$fp
	lw	$ra,40($sp)
	lw	$fp,36($sp)
	addu	$sp,$sp,48
	j	$ra
	.end	checkOption
	.size	checkOption, .-checkOption
	.align	2
	.globl	swapChars
	.ent	swapChars
swapChars:
	.frame	$fp,24,$ra		# vars= 8, regs= 2/0, args= 0, extra= 8
	.mask	0x50000000,-4
	.fmask	0x00000000,0
	.set	noreorder
	.cpload	$t9
	.set	reorder
	subu	$sp,$sp,24
	.cprestore 0
	sw	$fp,20($sp)
	sw	$gp,16($sp)
	move	$fp,$sp
	sw	$a0,24($fp)
	sw	$a1,28($fp)
	sw	$a2,32($fp)
	lw	$v1,28($fp)
	lw	$v0,32($fp)
	bne	$v1,$v0,$L24
	b	$L23
$L24:
	lw	$v1,24($fp)
	lw	$v0,32($fp)
	addu	$v0,$v1,$v0
	lbu	$v0,0($v0)
	sb	$v0,8($fp)
	lw	$v1,24($fp)
	lw	$v0,32($fp)
	addu	$a0,$v1,$v0
	lw	$v1,24($fp)
	lw	$v0,28($fp)
	addu	$v0,$v1,$v0
	lbu	$v0,0($v0)
	sb	$v0,0($a0)
	lw	$v1,24($fp)
	lw	$v0,28($fp)
	addu	$v1,$v1,$v0
	lbu	$v0,8($fp)
	sb	$v0,0($v1)
$L23:
	move	$sp,$fp
	lw	$fp,20($sp)
	addu	$sp,$sp,24
	j	$ra
	.end	swapChars
	.size	swapChars, .-swapChars
	.align	2
	.globl	reverseString
	.ent	reverseString
reverseString:
	.frame	$fp,56,$ra		# vars= 16, regs= 3/0, args= 16, extra= 8
	.mask	0xd0000000,-8
	.fmask	0x00000000,0
	.set	noreorder
	.cpload	$t9
	.set	reorder
	subu	$sp,$sp,56
	.cprestore 16
	sw	$ra,48($sp)
	sw	$fp,44($sp)
	sw	$gp,40($sp)
	move	$fp,$sp
	sw	$a0,56($fp)
	lw	$a0,56($fp)
	la	$t9,strlen
	jal	$ra,$t9
	sw	$v0,24($fp)
	lw	$v0,24($fp)
	addu	$v0,$v0,1
	move	$a0,$v0
	la	$t9,malloc
	jal	$ra,$t9
	sw	$v0,28($fp)
	lw	$a0,28($fp)
	lw	$a1,56($fp)
	la	$t9,strcpy
	jal	$ra,$t9
	sw	$zero,32($fp)
	lw	$v0,24($fp)
	addu	$v0,$v0,-2
	sw	$v0,36($fp)
$L26:
	lw	$v0,32($fp)
	lw	$v1,36($fp)
	slt	$v0,$v0,$v1
	bne	$v0,$zero,$L28
	b	$L27
$L28:
	addu	$v1,$fp,32
	lw	$v0,0($v1)
	move	$a1,$v0
	addu	$v0,$v0,1
	sw	$v0,0($v1)
	addu	$v1,$fp,36
	lw	$v0,0($v1)
	move	$a2,$v0
	addu	$v0,$v0,-1
	sw	$v0,0($v1)
	lw	$a0,28($fp)
	la	$t9,swapChars
	jal	$ra,$t9
	b	$L26
$L27:
	lw	$v0,28($fp)
	move	$sp,$fp
	lw	$ra,48($sp)
	lw	$fp,44($sp)
	addu	$sp,$sp,56
	j	$ra
	.end	reverseString
	.size	reverseString, .-reverseString
	.align	2
	.globl	readFromFile
	.ent	readFromFile
readFromFile:
	.frame	$fp,64,$ra		# vars= 24, regs= 3/0, args= 16, extra= 8
	.mask	0xd0000000,-8
	.fmask	0x00000000,0
	.set	noreorder
	.cpload	$t9
	.set	reorder
	subu	$sp,$sp,64
	.cprestore 16
	sw	$ra,56($sp)
	sw	$fp,52($sp)
	sw	$gp,48($sp)
	move	$fp,$sp
	sw	$a0,64($fp)
	li	$v0,30			# 0x1e
	sw	$v0,24($fp)
	lw	$v0,24($fp)
	addu	$v0,$v0,2
	move	$a0,$v0
	la	$t9,malloc
	jal	$ra,$t9
	sw	$v0,28($fp)
	sw	$zero,32($fp)
	sb	$zero,36($fp)
	sw	$zero,40($fp)
$L30:
	lw	$a0,64($fp)
	la	$t9,fgetc
	jal	$ra,$t9
	sb	$v0,36($fp)
	lbu	$v0,36($fp)
	sll	$v0,$v0,24
	sra	$v1,$v0,24
	li	$v0,-1			# 0xffffffffffffffff
	bne	$v1,$v0,$L32
	b	$L31
$L32:
	lw	$v0,40($fp)
	addu	$v0,$v0,1
	sw	$v0,40($fp)
	lw	$v0,40($fp)
	addu	$v1,$v0,1
	lw	$v0,24($fp)
	bne	$v1,$v0,$L33
	lw	$v0,24($fp)
	sll	$v0,$v0,1
	sw	$v0,24($fp)
	lw	$a0,28($fp)
	lw	$a1,24($fp)
	la	$t9,realloc
	jal	$ra,$t9
	sw	$v0,32($fp)
	lw	$v0,32($fp)
	sw	$v0,28($fp)
$L33:
	lw	$v1,28($fp)
	lw	$v0,40($fp)
	addu	$v0,$v1,$v0
	addu	$v1,$v0,-1
	lbu	$v0,36($fp)
	sb	$v0,0($v1)
	lb	$v1,36($fp)
	li	$v0,10			# 0xa
	bne	$v1,$v0,$L30
$L31:
	addu	$a1,$fp,40
	lw	$v1,0($a1)
	move	$a0,$v1
	lw	$v0,28($fp)
	addu	$v0,$a0,$v0
	sb	$zero,0($v0)
	addu	$v1,$v1,1
	sw	$v1,0($a1)
	lw	$v1,40($fp)
	li	$v0,1			# 0x1
	bne	$v1,$v0,$L35
	lw	$a0,28($fp)
	la	$t9,free
	jal	$ra,$t9
	sw	$zero,44($fp)
	b	$L29
$L35:
	lw	$a0,28($fp)
	lw	$a1,40($fp)
	la	$t9,realloc
	jal	$ra,$t9
	sw	$v0,32($fp)
	lw	$v0,32($fp)
	sw	$v0,44($fp)
$L29:
	lw	$v0,44($fp)
	move	$sp,$fp
	lw	$ra,56($sp)
	lw	$fp,52($sp)
	addu	$sp,$sp,64
	j	$ra
	.end	readFromFile
	.size	readFromFile, .-readFromFile
	.rdata
	.align	2
$LC6:
	.ascii	"%s\000"
	.text
	.align	2
	.globl	reverseFile
	.ent	reverseFile
reverseFile:
	.frame	$fp,48,$ra		# vars= 8, regs= 3/0, args= 16, extra= 8
	.mask	0xd0000000,-8
	.fmask	0x00000000,0
	.set	noreorder
	.cpload	$t9
	.set	reorder
	subu	$sp,$sp,48
	.cprestore 16
	sw	$ra,40($sp)
	sw	$fp,36($sp)
	sw	$gp,32($sp)
	move	$fp,$sp
	sw	$a0,48($fp)
	sw	$zero,24($fp)
	lw	$a0,48($fp)
	la	$t9,readFromFile
	jal	$ra,$t9
	sw	$v0,28($fp)
$L37:
	lw	$v0,28($fp)
	bne	$v0,$zero,$L39
	b	$L36
$L39:
	lw	$a0,28($fp)
	la	$t9,reverseString
	jal	$ra,$t9
	sw	$v0,24($fp)
	la	$a0,$LC6
	lw	$a1,24($fp)
	la	$t9,printf
	jal	$ra,$t9
	lw	$a0,28($fp)
	la	$t9,free
	jal	$ra,$t9
	lw	$a0,24($fp)
	la	$t9,free
	jal	$ra,$t9
	lw	$a0,48($fp)
	la	$t9,readFromFile
	jal	$ra,$t9
	sw	$v0,28($fp)
	b	$L37
$L36:
	move	$sp,$fp
	lw	$ra,40($sp)
	lw	$fp,36($sp)
	addu	$sp,$sp,48
	j	$ra
	.end	reverseFile
	.size	reverseFile, .-reverseFile
	.rdata
	.align	2
$LC7:
	.ascii	"r\000"
	.text
	.align	2
	.globl	main
	.ent	main
main:
	.frame	$fp,56,$ra		# vars= 16, regs= 3/0, args= 16, extra= 8
	.mask	0xd0000000,-8
	.fmask	0x00000000,0
	.set	noreorder
	.cpload	$t9
	.set	reorder
	subu	$sp,$sp,56
	.cprestore 16
	sw	$ra,48($sp)
	sw	$fp,44($sp)
	sw	$gp,40($sp)
	move	$fp,$sp
	sw	$a0,56($fp)
	sw	$a1,60($fp)
	sw	$zero,24($fp)
	lw	$v1,56($fp)
	li	$v0,1			# 0x1
	bne	$v1,$v0,$L41
	la	$a0,__sF
	la	$t9,reverseFile
	jal	$ra,$t9
	sw	$zero,32($fp)
	b	$L40
$L41:
	lw	$v1,56($fp)
	li	$v0,2			# 0x2
	bne	$v1,$v0,$L42
	lw	$v0,60($fp)
	addu	$v0,$v0,4
	lw	$a0,0($v0)
	la	$t9,checkOption
	jal	$ra,$t9
	beq	$v0,$zero,$L42
	sw	$zero,32($fp)
	b	$L40
$L42:
	li	$v0,1			# 0x1
	sw	$v0,28($fp)
$L44:
	lw	$v0,28($fp)
	lw	$v1,56($fp)
	sltu	$v0,$v0,$v1
	bne	$v0,$zero,$L47
	b	$L45
$L47:
	lw	$v0,28($fp)
	sll	$v1,$v0,2
	lw	$v0,60($fp)
	addu	$v0,$v1,$v0
	lw	$a0,0($v0)
	la	$a1,$LC7
	la	$t9,fopen
	jal	$ra,$t9
	sw	$v0,24($fp)
	lw	$a0,24($fp)
	la	$t9,reverseFile
	jal	$ra,$t9
	lw	$a0,24($fp)
	la	$t9,fclose
	jal	$ra,$t9
	lw	$v0,28($fp)
	addu	$v0,$v0,1
	sw	$v0,28($fp)
	b	$L44
$L45:
	sw	$zero,32($fp)
$L40:
	lw	$v0,32($fp)
	move	$sp,$fp
	lw	$ra,48($sp)
	lw	$fp,44($sp)
	addu	$sp,$sp,56
	j	$ra
	.end	main
	.size	main, .-main
	.ident	"GCC: (GNU) 3.3.3 (NetBSD nb3 20040520)"

\end{verbatim}
\end{document}
